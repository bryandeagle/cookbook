\documentclass{cookbook}
\begin{document}



\recipe{Belgian Beef Stew}

\begin{ingredients}
    \item 2 lb. beef chuck, cut into 2 in. x \sfrac{1}{2} in. slices
    \item \sfrac{1}{4} cup flour
    \item 4 tbsp. unsalted butter
    \item 4 slices bacon, finely chopped
    \item 6 cloves garlic, finely chopped
    \item 3 medium yellow onions, diced
    \item 2 cups Belgian-style ale
    \item 1 cup beef stock
    \item 2 tbsp. dark brown sugar
    \item 2 tbsp. apple cider vinegar
    \item 3 sprigs thyme
    \item 3 sprigs parsley
    \item 2 sprigs tarragon
    \item 1 bay leaf
\end{ingredients}

Season beef with salt and pepper in a bowl. Add flour and toss to coat. Heat 2 tbsp. butter in a 6 qt. Dutch oven over medium-high heat. Working in batches, add beef. Cook, turning, until browned, about 8 minutes. Transfer to a plate and set aside.

Add bacon and cook until its fat renders, about 8 minutes. Add remaining butter, garlic, and onions. Cook until caramelized, about 30 minutes.

Add half the beer. Cook, scraping bottom of pot, until slightly reduced, about 4 minutes. Return beef to pot with remaining beer, stock, sugar, vinegar, thyme, parsley, tarragon, bay leaf, and salt and pepper. Boil. Reduce heat to medium-low. Cook, covered, until beef is tender, about 1 ½ hours.



\recipe{Chicken Enchilada Bowl}

\begin{ingredients}
    \item 1 lb. boneless skinless chicken thighs
    \item \sfrac{3}{4} cup red enchilada sauce
    \item \sfrac{1}{4} cup water
    \item \sfrac{1}{4} cup chopped onion
    \item 4 oz. can diced green chiles
    \item 1 whole avocado, diced
    \item 1 cup shredded cheese
    \item \sfrac{1}{4} cup chopped pickled jalapenos
    \item \sfrac{1}{2} cup sour cream
    \item 1 roma tomato, chopped
\end{ingredients}

In a pot or dutch oven over medium heat sear chicken thighs until lightly brown.

Pour in enchilada sauce and water then add onion and green chiles. Reduce heat to a simmer and cover. Cook chicken for 17-25 minutes or until chicken is tender and fully cooked through.

Careully remove the chicken and place onto a work surface. Chop or shred chicken then add it back into the pot. Let the chicken simmer uncovered for an additional 10 minutes to absorb flavor and allow the sauce to reduce a little.

To Serve, top with avocado, cheese, jalapeno, sour cream, tomato, and any other desired toppings. Feel free to customize these to your preference.



\recipe{Instant Pot Beef Stroganoff}

\begin{ingredients}
    \item 1 lb. ground beef (or ground turkey or ground pork)
    \item 1 small onion, chopped
    \item 1 clove garlic, minced
    \item 8 oz. white button mushrooms, sliced
    \item 1 tbsp. all-purpose flour
    \item 1 can cream of mushroom soup (or homemade)
    \item 2 \sfrac{1}{4} cups low-sodium beef broth
    \item 8 oz. wide egg noodle
    \item \sfrac{1}{3} cup sour cream
    \item 2 tbsp. fresh parsley leaves, chopped
\end{ingredients}

Turn Instant Pot to saute. Once the pot is hot, spray the bottom with cooking spray (or add a little oil), and add ground beef. Brown the meat, breaking it into small pieces as it cooks, and season it with salt and pepper. Remove any excess grease.

Add onion, garlic and sliced mushrooms and saute for 1 minute. Add the flour, homemade or canned cream of mushroom soup, beef broth, and noodles and stir to combine.

Secure instant pot lid, and turn valve to sealed position. Cook on Manual Setting or High Pressure Setting for 2 minutes.

When pressure cooking is complete, use a quick release. (If liquid sprays from the knob, close knob, wait 30 seconds then release pressure again. You may also turn knob half way so only a small amount of steam is being released.)

Carefully remove the lid and stir mixture. Allow the mixture to cool for a few minutes, then stir in sour cream, parsley, additional salt and pepper to taste.

\end{document}
